\chapter{Planteamiento del problema}

\section{El problema}

La manufactura aditiva, también conocida como impresión 3D es una técnica de prototipado rápido que, a pesar de estar en su infancia, está madurando muy rápidamente, con un potencial aparentemente ilimitado. Con su capacidad de reproducir objetos 3D arqueológicos, superficies matemáticas complejas, hasta prótesis médicas, la tecnología tiene un futuro especialmente prometedor para la ciencia, la educación y el desarrollo sostenible (\cite{LowCost3DP}). Incluso algunos autores consideran a esta técnica una revolución en el mundo de la manufactura (\cite{AManufacturingRev}).

\newpage
\section{Objetivos de la investigación}

\subsection{Objetivo General}
Desarrollar un ...

\subsection{Objetivos Específicos}
\begin{itemize}
    \item Estudiar ...
    \item Comparar ...
	\item Realizar ...
	\item Fabricar ...
\end{itemize}

\newpage
\section{Justificación de la investigación}

La investigación tiene como propósito ...

\newpage
\section{Alcances y limitaciones}

\subsection{Alcances}

\begin{itemize}
	\item Alcance 1.
	\item Alcance 2.
	\item Alcance 3.
	\item Alcance 4.
\end{itemize}

\subsection{Limitaciones}

\begin{itemize}
	\item Limitación 1.
	\item Limitación 2.
	\item Limitación 3.
\end{itemize}
